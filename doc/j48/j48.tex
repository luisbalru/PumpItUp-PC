\begin{frame}{Primera aproximación}
  \metroset{block=fill}
  \begin{block}{Eliminación de variables}
    \begin{itemize}
    \item scheme\_name: 1/2 de valores perdidos
    \item Variables categóricas con más de 100 características
    \item recorded\_by: una única categoría
    \item region\_code: Correlada con district\_code
    \item construction\_year: Correlada con gps\_height y valores perdidos
      (valores 0 en esta columna no tienen sentido)
    \end{itemize}
  \end{block}
  \begin{block}{Imputación de valores perdidos}
    \begin{itemize}
    \item Variables numéricas: valor medio
    \item Variables categóricas: valor modal
    \end{itemize}
  \end{block}
  \begin{block}{Resultado}
    0.7677
  \end{block}
\end{frame}

\begin{frame}{Mejoras a dicha aproximación}
  \metroset{block=fill}
  \begin{block}{Eliminación de filas con missing values}
    \begin{itemize}
    \item Quedan 49841 filas en el conjunto
    \item Precisión: 0.7328 $\rightarrow$ se descarta la vía
    \end{itemize}
  \end{block}
  \begin{block}{Imputación en train por clase}
    \begin{itemize}
    \item En lugar de la media y la moda globales, se imputa por media
      y mediana de la clase
    \item En test, seguimos imputando igual
    \item Precisión: 0.7677 $\rightarrow$ no mejora el resultado, se
      descarta
    \end{itemize}
  \end{block}
  \begin{block}{Creación de una nueva categoría para los valores perdidos}
    \begin{itemize}
    \item En las variables categóricas se añade una categoría nueva,
      que representa el valor perdido.
    \item Precisión: 0.7385 $\rightarrow$ se descarta la vía
    \end{itemize}
  \end{block}
\end{frame}

\begin{frame}{Segundo estudio - importancia de las variables}
  \metroset{block=fill}
  \begin{block}{Medición de la importancia de las variables}
    Para las variables numéricas con valores perdidos en
    entrenamiento, se computa el valor de dichas variables con la
    media de los valores de la clase, dejando el resto de variables
    imputadas de forma poco inteligente.\\\\

    Tras esto, se clasifica el conjunto de entrenamiento con
    validación cruzada.
  \end{block}
  \begin{block}{Intuición}
    Las variables que produzcan una mejora importante en el resultado
    contendrán información relevante para la solución del problema.
  \end{block}
\end{frame}

\begin{frame}{Segundo estudio - importancia de las variables}
  \metroset{block=fill}
  \begin{block}{Resultados}
    Mejora escasa o nula en todas las variables.\\\\

    Imputación correcta de \texttt{construction\_year} $\rightarrow$
    Mejora en la CV de 0.78 a 0.84.\\\\

    La variable \texttt{construction\_year} parece muy relevante a
    la hora de establecer la clasificación
  \end{block}
\end{frame}
